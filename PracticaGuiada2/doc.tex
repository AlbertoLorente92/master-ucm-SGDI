\documentclass{article}
\usepackage{graphicx}
\usepackage[spanish]{babel}
\usepackage[utf8]{inputenc}
\usepackage{listings}
\usepackage{caption} 
\usepackage[T1]{fontenc}

\captionsetup{skip=-30pt}

\pagenumbering{gobble}


\begin{document}

\title{Sistemas de Gestión de Datos y de la Información\\Práctica Guiada 2.}
\author{Alberto Lorente y Hristo Ivanov}
\maketitle

  \newpage
  \section{Pregunta 1.}
    !!! TODO añadir explicación
    \begin{lstlisting}[numbers=left,frame=single]
"winningPlan" : {
  "stage" : "SORT",
  "sortPattern" : {
    "username" : 1
  },
  "inputStage" : {
    "stage" : "SORT_KEY_GENERATOR",
    "inputStage" : {
      "stage" : "COLLSCAN",
      "filter" : {
        "year" : {
          "$eq" : 1980
        }
      },
      "direction" : "forward"
    }
  }
}
    \end{lstlisting}


  \newpage
  \section{Pregunta 2.}
    !!! TODO añadir explicación
    \begin{lstlisting}[numbers=left,frame=single]
"winningPlan" : {
  "stage" : "SORT",
  "sortPattern" : {
    "username" : 1
  },
  "inputStage" : {
    "stage" : "KEEP_MUTATIONS",
    "inputStage" : {
      "stage" : "SORT_KEY_GENERATOR",
      "inputStage" : {
        "stage" : "FETCH",
        "inputStage" : {
          "stage" : "IXSCAN",
          "keyPattern" : {
            "year" : 1
          },
          "indexName" : "year_1",
          "isMultiKey" : false,
          "isUnique" : false,
          "isSparse" : false,
          "isPartial" : false,
          "indexVersion" : 1,
          "direction" : "forward",
          "indexBounds" : {
            "year" : [
              "[1980.0, 1980.0]"
            ]
          }
        }
      }
    }
  }
},
    \end{lstlisting}


  \newpage
  \section{Pregunta 3.}
    !!! TODO añadir explicación. Quitar las ultimas lineas y añadirlas en la explicación.
    \begin{lstlisting}[numbers=left,frame=single]
"winningPlan" : {
  "stage" : "FETCH",
  "filter" : {
    "like" : {
      "$eq" : "deportes"
    }
  },
  "inputStage" : {
    "stage" : "IXSCAN",
    "keyPattern" : {
      "year" : 1
    },
    "indexName" : "year_1",
    "isMultiKey" : false,
    "isUnique" : false,
    "isSparse" : false,
    "isPartial" : false,
    "indexVersion" : 1,
    "direction" : "forward",
    "indexBounds" : {
      "year" : [
        "[MinKey, MaxKey]"
      ]
    }
  }
},

"totalKeysExamined" : 20000,
"totalDocsExamined" : 20000,
En el IXSCAN.
"keysExamined" : 20000,
En el FETCH.
"docsExamined" : 20000,

    \end{lstlisting}


  \newpage
  \section{Pregunta 4.}
    !!! TODO añadir explicación
    \begin{lstlisting}[numbers=left,frame=single]
"winningPlan" : {
  "stage" : "PROJECTION",
  "transformBy" : {
    "_id" : 1
  },
  "inputStage" : {
    "stage" : "IXSCAN",
    "keyPattern" : {
      "year" : 1,
      "_id" : 1
    },
    "indexName" : "year_1__id_1",
    "isMultiKey" : false,
    "isUnique" : false,
    "isSparse" : false,
    "isPartial" : false,
    "indexVersion" : 1,
    "direction" : "forward",
    "indexBounds" : {
      "year" : [
        "[1980.0, 1980.0]"
      ],
      "_id" : [
        "[MinKey, MaxKey]"
      ]
    }
  }
},

"totalKeysExamined" : 164,
"totalDocsExamined" : 0,

"rejectedPlans" : [
  {
    "stage" : "PROJECTION",
    "transformBy" : {
      "_id" : 1
    },
    "inputStage" : {
      "stage" : "FETCH",
      "inputStage" : {
        "stage" : "IXSCAN",
        "keyPattern" : {
          "year" : 1
        },
        "indexName" : "year_1",
        "isMultiKey" : false,
        "isUnique" : false,
        "isSparse" : false,
        "isPartial" : false,
        "indexVersion" : 1,
        "direction" : "forward",
        "indexBounds" : {
          "year" : [
            "[1980.0, 1980.0]"
          ]
        }
      }
    }
  }
]
},
    \end{lstlisting}

  \section{Pregunta 5.}
    El servidor primario es el 27102. Al ser este el primero en formar parte
    del \emph{Replica Set}. Los demás al ser añadidos después, han tomado a
    este como primario.

  \section{Pregunta 6.}
    Sí. Al perder conexión con el servidor primario estos han realizado una
    votación para elegir el nuevo servidor primario. En este caso el nuevo
    servidor primario ha sido el 27101. El \texttt{uptime} tiene gran peso en
    la toma de esta decisión, pero también se toman otros factores en
    consideración.

  \section{Pregunta 7.}
    En nuestro caso el 27103 sigue vivo, en estado de segundario. El servidor
    27103 sigue como segundario porque no sabe si el servidor primario sigue
    vivo o simplemente no esta alcanzable. En cambio en la configuración
    anterior teniamos dos servidores segundarios que no eran capaces de
    alcanzar el primario. 

  \section{Pregunta 8.}
    En el primero.

  \section{Pregunta 9.}
  Chunks total : 14.
  \begin{itemize}
    \item Chunks 27101 : 5
    \item Chunks 27102 : 5
    \item Chunks 27103 : 4
  \end{itemize}
  Los rangos pueden apreciar se en la siguiente lista:
  \begin{lstlisting}[numbers=left, basicstyle=\tiny]
{ "username" : { "$minKey" : 1 } } -->> { "username" : "DnGYx" } on : shard0001 Timestamp(2, 0) 
{ "username" : "DnGYx" } -->> { "username" : "HZGYA" } on : shard0002 Timestamp(3, 0) 
{ "username" : "HZGYA" } -->> { "username" : "LMwRR" } on : shard0001 Timestamp(4, 0) 
{ "username" : "LMwRR" } -->> { "username" : "PEeab" } on : shard0002 Timestamp(5, 0) 
{ "username" : "PEeab" } -->> { "username" : "Srtfj" } on : shard0001 Timestamp(6, 0) 
{ "username" : "Srtfj" } -->> { "username" : "WdqWk" } on : shard0002 Timestamp(7, 0) 
{ "username" : "WdqWk" } -->> { "username" : "aSneZ" } on : shard0001 Timestamp(8, 0) 
{ "username" : "aSneZ" } -->> { "username" : "eHexD" } on : shard0002 Timestamp(9, 0) 
{ "username" : "eHexD" } -->> { "username" : "hxwlf" } on : shard0001 Timestamp(10, 0) 
{ "username" : "hxwlf" } -->> { "username" : "lnVtN" } on : shard0000 Timestamp(10, 1) 
{ "username" : "lnVtN" } -->> { "username" : "peFcR" } on : shard0000 Timestamp(1, 10) 
{ "username" : "peFcR" } -->> { "username" : "tVaHf" } on : shard0000 Timestamp(1, 11) 
{ "username" : "tVaHf" } -->> { "username" : "xLoBa" } on : shard0000 Timestamp(1, 12) 
{ "username" : "xLoBa" } -->> { "username" : { "$maxKey" : 1 } } on : shard0000 Timestamp(1, 13) 
  \end{lstlisting}


  \newpage
  \section{Pregunta 10.}
    \begin{itemize}
      \item \texttt{e.find({'\_id':67})}
        \begin{itemize}
          \item Número de shards: X
          \item Resultados devueltos por shard:
            \begin{itemize}
              \item Shard 27101 : X
              \item Shard 27102 : X
              \item Shard 27103 : X
            \end{itemize}
          \item Tipo de busqueda por shard:
            \begin{itemize}
              \item Shard 27101 : X
              \item Shard 27102 : X
              \item Shard 27103 : X
            \end{itemize}
          \item Número total de documentos examinados : X
        \end{itemize}
      \item \texttt{e.find({'year':1997})}
        \begin{itemize}
          \item Número de shards: X
          \item Resultados devueltos por shard:
            \begin{itemize}
              \item Shard 27101 : X
              \item Shard 27102 : X
              \item Shard 27103 : X
            \end{itemize}
          \item Tipo de busqueda por shard:
            \begin{itemize}
              \item Shard 27101 : X
              \item Shard 27102 : X
              \item Shard 27103 : X
            \end{itemize}
          \item Número total de documentos examinados : X
        \end{itemize}
      \item \texttt{e.find({'username':'Aaron'})}
        \begin{itemize}
          \item Número de shards: X
          \item Resultados devueltos por shard:
            \begin{itemize}
              \item Shard 27101 : X
              \item Shard 27102 : X
              \item Shard 27103 : X
            \end{itemize}
          \item Tipo de busqueda por shard:
            \begin{itemize}
              \item Shard 27101 : X
              \item Shard 27102 : X
              \item Shard 27103 : X
            \end{itemize}
          \item Número total de documentos examinados : X
        \end{itemize}
    \end{itemize}



\end{document}
