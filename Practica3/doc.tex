\documentclass{article}
\usepackage{graphicx}
\usepackage[spanish]{babel}
\usepackage[utf8]{inputenc}
\usepackage{listings}
\usepackage{caption} 

\captionsetup{skip=-30pt}

\pagenumbering{gobble}


\begin{document}

\title{Sistemas de Gestión de Datos y de la Información. Práctica 3. \\ Esquema implícito.}
\author{Alberto Lorente y Hristo Ivanov}
\maketitle

  \begin{abstract}
     Selección del esquema implícito para la base de datos de MongoDB.
  \end{abstract}

  \section{Elección}
    Se ha elegido una estructura de 3 colecciones. Estas colecciones seran \emph{Usuarios, Preguntas y Contestaciones}
    \par
    La colección \texttt{Usuarios} contendrá los siguientes atributos:   
    \begin{itemize}  
        \item _id: correspondiente al alias del usuario 
        \item nombre
        \item apellidos
        \item experiencia: lista de conocimientos del usuario
        \item fecha: fecha de ingreso en la web
        \item direccion
        \begin{enumerate}
          \item pais
          \item ciudad
          \item cp: codigo postal
        \end{enumerate}
    \end{itemize}
    \par
    La colección \texttt{Preguntas} contendrá los siguientes atributos:   
    \begin{itemize}  
        \item _id: autoasignado por la base de datos 
        \item titulo
        \item texto
        \item fecha: fecha en la que se colgo la pregunta
        \item tags: lista de etiquetas de la pregunta
        \item idusuario: alias del usuario que cuelga la pregunta
    \end{itemize}
    \par
    La colección \texttt{Contestaciones} contendrá los siguientes atributos:   
    \begin{itemize}  
        \item _id: autoasignado por la base de datos 
        \item texto
        \item fecha: fecha en la que se colgo la pregunta
        \item valoracion
        \begin{enumerate}
          \item fecha: fecha de la valoracion
          \item nota: goor or bad
          \item idusuario: alias del usuario que evalua la respuesta
        \end{enumerate}
        \item idusuario: alias del usuario que cuelga la respuesta
        \item idpregunta: id de la pregunta que responde este documento
        \item comentario
        \begin{enumerate}
          \item fecha: fecha del comentario sobre la respuesta
          \item texto
          \item idusuario: alias del usuario que comenta la respuesta
        \end{enumerate}
    \end{itemize}
    \par
    Tenemos 4 posibles documentos \emph{Usuarios, Preguntas, Contestaciones y Comentarios}
    y se ha decidido anidar los comentarios dentro de las Contestaciones. 

  \section{Argumentación}
    Existen numerosas posibilidades para hacer el esquema implícito de la base de datos.
    Sin embargo se ha elegido esta ya que es en nuestra opinión la más fácil de manejar y
    la que mejor se adapta a todas las consultas que se van a realizar.
    \par
    A continuación se listan las consultas y se marcan como buenas o malas según lo adecuado
    con el esquema implícito elegido
    \par
    \begin{enumerate}
      \item Añadir un usuario. \\
      Solo añadir otro documento a la colección de usuario. Buena.
      \item Actualizar un usuario. \\
      Solo modifica un documento en la colección de usuario. Buena.
      \item Añadir una pregunta. \\
      Solo añadir otro documento a la colección de preguntas. Buena.      
      \item Añadir una respuesta a una pregunta. \\
      Solo añadir otro documento a la colección de contestaciones. Buena.
      \item Comentar una respuesta. \\
      Solo modificar un documento en la colección de contestaciones. Buena.
      \item Puntuar una respuesta. \\
      Solo modificar un documento en la colección de contestaciones. Buena.
      \item Modificar una puntuación de buena a mala o viceversa. \\
      Solo modificar un documento en la colección de contestaciones. Buena.
      \item Borrar una pregunta junto con todas sus respuestas, comentarios y puntuaciones. \\
      Borrar un documento de la colección de preguntas y \emph{n} de contestaciones.
      Esto supone 2 accesos a la base de datos, lo que puede provocar alguna clase de inconsistencia
      por la ejecución al mismo tiempo de otra operación. Sin embargo facilita el manejo de
      la base de datos. Buena
      \item Visualizar una determinada pregunta junto con todas sus contestaciones
      y comentarios. A su vez las contestaciones vendrán acompañadas de su
      número de puntuaciones buenas y malas. \\
      El mismo caso que el apartado anterior. Buena.
      \item Buscar preguntas con unos determinados tags y mostrar su título, su autor
      y su número de contestaciones. \\
      En este caso debemos buscar todas las preguntas con ciertos tags y posteriormente
      todas las respuestas asociadas a dichas preguntas. Esto supone 2 consultas como 
      en el apartado anterior. Buena.
      \item Ver todas las preguntas o respuestas generadas por un determinado usuario. \\
      Este es un caso semejante al anterior. Buena.
      \item Ver todas las puntuaciones de un determinado usuario ordenadas por fecha.
      Este listado debe contener el título de la pregunta original cuya respuesta
      se puntuó. \\
      Semejante al apartado anterior se debe ahcer 2 accesos a la base de datos para 
      encontrar las valoraciones y añadirles el título de la pregunta. Buena.
      \item Ver todos los datos de un usuario.\\
      Solo se busca en la colección de usuarios. Buena.
      \item Obtener los alias de los usuarios expertos en un determinado tema.\\
      Solo se busca en la colección de usuarios. Buena.
      \item Visualizar las n preguntas más actuales ordenadas por fecha, incluyendo
      el número de contestaciones recibidas.\\
      Supone 2 accesos a la base de datos. Uno primero para buscar las preguntas más actuales
      y otro para añadir el número de contestaciones que tienen cada una. Buena.
      \item Ver n preguntas sobre un determinado tema, ordenadas de mayor a menor
      por número de contestaciones recibidas.\\
      Se accede 2 veces a la base de datos preguntando sobre un tema a la colección de
      preguntas y sobre el número de contestaciones a la colección de contestaciones. Esto 
      ocasiona una dificultad dado que no podemos extraer de forma ordenada por el número de
      contestaciones recibidas en una pregunta. Mala
    \end{enumerate}

    


\end{document}
