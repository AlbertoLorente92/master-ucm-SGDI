\documentclass{article}
\usepackage{graphicx}
\usepackage[spanish]{babel}
\usepackage[utf8]{inputenc}
\usepackage{listings}
\usepackage{caption} 

\captionsetup{skip=-30pt}

\pagenumbering{gobble}


\begin{document}

\title{Sistemas de Gestión de Datos y de la Información. Práctica 3. \\ Esquema implícito.}
\author{Alberto Lorente y Hristo Ivanov}
\maketitle

  \begin{abstract}
     Selección del esquema implícito para la base de datos de MongoDB.
  \end{abstract}

  \section{Elección}
    Se ha elegido una estructura de 3 colecciones. Estas colecciones seran \emph{Usuarios, Preguntas y Contestaciones}
    \par
    La colección \texttt{Usuarios} contendrá los siguientes atributos:   
    \begin{itemize}  
        \item \_id: correspondiente al alias del usuario 
        \item nombre
        \item apellidos
        \item experiencia: lista de conocimientos del usuario
        \item fecha: fecha de ingreso en la web
        \item direccion
        \begin{enumerate}
          \item pais
          \item ciudad
          \item cp: codigo postal
        \end{enumerate}
    \end{itemize}
    \par
    La colección \texttt{Preguntas} contendrá los siguientes atributos:   
    \begin{itemize}  
        \item \_id: autoasignado por la base de datos 
        \item titulo
        \item texto
        \item fecha: fecha en la que se colgo la pregunta
        \item tags: lista de etiquetas de la pregunta
        \item idusuario: alias del usuario que cuelga la pregunta
    \end{itemize}
    \par
    La colección \texttt{Contestaciones} contendrá los siguientes atributos:   
    \begin{itemize}  
        \item \_id: autoasignado por la base de datos 
        \item texto
        \item fecha: fecha en la que se colgo la pregunta
        \item valoracion
        \begin{enumerate}
          \item fecha: fecha de la valoracion
          \item nota: goor or bad
          \item idusuario: alias del usuario que evalua la respuesta
        \end{enumerate}
        \item idusuario: alias del usuario que cuelga la respuesta
        \item idpregunta: id de la pregunta que responde este documento
        \item comentario
        \begin{enumerate}
          \item fecha: fecha del comentario sobre la respuesta
          \item texto
          \item idusuario: alias del usuario que comenta la respuesta
        \end{enumerate}
    \end{itemize}
    \par
    Tenemos 4 posibles documentos \emph{Usuarios, Preguntas, Contestaciones y Comentarios}
    y se ha decidido anidar los comentarios dentro de las Contestaciones. 

  \section{Argumentación}
    A continuación vamos a analizar como se adapta el esquema propuesto a las
    operaciones que queremos realizar. Vamos a valorar estas como
    \emph{buenas}, \emph{normales} o \emph{malas}. Los factores que tendremos
    en cuenta son varios. Los más importantes son consistencia de los datos,
    simplicidad de la operación, número de accesos a la base de datos y
    eficiencia. También es importante recalcar que no utilizaramos el
    \texttt{aggregation framework} para la realización de estas operaciones.
    \par
    \begin{enumerate}
      \item Añadir un usuario. \\ 
        Añadir un documento a la colección de \texttt{Usuarios}. \textbf{Buena}.
      \item Actualizar un usuario. \\
        Modificar un documento en la colección de \texttt{Usuarios}.. \textbf{Buena}.
      \item Añadir una pregunta. \\
        Una consulta a \texttt{Usuarios} para comprobar si el usuarios existe(\texttt{idusuario}).\\  
        Añadir un documento a la colección de \texttt{Preguntas}. \textbf{Normal}.      
      \item Añadir una respuesta a una pregunta. \\
        Una consulta a \texttt{Usuarios} para comprobar si el usuarios existe(\texttt{idusuario}).\\  
        Añadir un documento a la colección de \texttt{Contestaciones}. \textbf{Normal}.
      \item Comentar una respuesta. \\
        Una consulta a \texttt{Usuarios} para comprobar si el usuarios existe(\texttt{idusuario}).\\  
        Modificar un documento en la colección de \texttt{Contestaciones}. \textbf{Normal}.
      \item Puntuar una respuesta. \\
        Una consulta a \texttt{Usuarios} para comprobar si el usuarios existe(\texttt{idusuario}).\\  
        Modificar un documento en la colección de \texttt{Contestaciones}. \textbf{Normal}.
      \item Modificar una puntuación de buena a mala o viceversa. \\
        Solo modificar un documento en la colección de \texttt{Contestaciones}. \textbf{Buena}.
      \item Borrar una pregunta junto con todas sus respuestas, comentarios y puntuaciones. \\ 
        Borrar un documento de la colección \texttt{Preguntas}. \\
        Borrar \emph{n} documento de la colección \texttt{Contestaciones}, que
        se consigue en una sola querry.  \textbf{Normal}
      \item Visualizar una determinada pregunta junto con todas sus contestaciones
        y comentarios. A su vez las contestaciones vendrán acompañadas de su
        número de puntuaciones buenas y malas. \\
        Una consulta a la colección \texttt{Preguntas}. \\
        Una consulta a la colección \texttt{Contestaciones}, en la que nos
        traemos \emph{n} documentos.  \textbf{Normal}
      \item Buscar preguntas con unos determinados tags y mostrar su título, su autor
        y su número de contestaciones. \\
        Una consulta a \texttt{Preguntas} para recuperar las preguntas. \\
        Una consulta por cada pregunta a \texttt{Contestaciones} para recuperar
        el número de contestaciones. \textbf{Mala}.
      \item Ver todas las preguntas o respuestas generadas por un determinado usuario. \\
        Dos consultas, una a \texttt{Preguntas} y la segunta a \texttt{Contestaciones}.
      \item Ver todas las puntuaciones de un determinado usuario ordenadas por fecha.
        Este listado debe contener el título de la pregunta original cuya respuesta
        se puntuó. \\ 
        Una consulta a \texttt{Contestaciones} para recuperar las puntuaciones. \\
        Una consulta por cada puntuación para recuperar los títulos de las
        preguntas. \textbf{Mala}.
      \item Ver todos los datos de un usuario.\\
        Una consulta a la colección \texttt{Usuarios}. \textbf{Buena}.
      \item Obtener los alias de los usuarios expertos en un determinado tema.\\
        Una consulta a la colección \texttt{Usuarios}. \textbf{Buena}.
      \item Visualizar las n preguntas más actuales ordenadas por fecha, incluyendo
        el número de contestaciones recibidas.\\ 
        Una consulta a \texttt{Preguntas} para recuperar las preguntas. \\
        Una consulta por cada pregunta a \texttt{Contestaciones} para recuperar
        el número de contestaciones. \textbf{Mala}.
      \item Ver n preguntas sobre un determinado tema, ordenadas de mayor a menor
        por número de contestaciones recibidas.\\
        Una consulta a \texttt{Preguntas} para recuperar las preguntas. \\
        Una consulta por cada pregunta a \texttt{Contestaciones} para recuperar
        el número de contestaciones. \textbf{Mala}.
    \end{enumerate}
    Como podemos ver nuestro esquema no es perfecto. Sin embargo al tener las
    preguntas y contestaciones en colecciones separadas este se adapta muy bien
    a las siguientes consideraciones que debemos tener en cuenta.
    \\ \par
    \emph{
      El número de contestaciones a una pregunta no está acotado.
    }
    \\ \par
    \emph{
      Deseamos que las contestaciones puedan tener contenido multimedia si es
      necesario, por lo que su tamaño puede ser grande.
    }

    


\end{document}
